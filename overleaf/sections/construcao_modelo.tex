% Referência - http://cs231n.github.io/convolutional-networks/

\section{Construção do Modelo}

As Redes Convolucionais Neurais - RCN, tomam vantagem do fato de que a sua entrada é uma imagem para limitar sensivelmente sua arquitetura. Que diferentemente de Redes Neurais Artificiais - RNA comuns, possuem neurônios organizados em três dimensões, largura, altura e profundidade. Além disso, neurônios de uma determinada camada estão conectados apenas a uma pequena região da camada anterior, e assim percebe-se que suas camadas são sequenciais.

As arquiteturas de RCN possuem três tipos principais de camadas, são elas: Convolucional, \textit{Pooling} e RNA completamente conectada. Que juntas são empilhadas para formar uma arquitetura completa de RCN. Cada camada tem sua função específica e bem definida na arquitetura. As camadas Convolucionais tem por finalidade extrair características da imagem, como bordas, contornos, texturas e etc. As camadas de \textit{Pooling} são as responsáveis por afunilar as informações das camada anterior, tanto para reduzir o custo computacional dos cálculos, quanto para diminuir o número de características a serem processadas para a próxima camada, tentando assim diminuir o conjunto de características relevantes da imagem. A última camada, a de RNA completamente conectada, consiste da generalização das características extraídas pelas camadas anteriores.

% Tópicos
% 1. Funções de ativação
%     1.1 Nas camadas convolucionais
%     1.2 Nas camadas de FC
% 2. Inicialização dos pesos das camadas convolucionais
% 3. Tamanho dos kerneis das camadas convolucionais
% 4. Tipos de Pooling
% 5. Quantidade de camadas antes do pooling
%     5.1 quanto maior a quantidade, mais aprende, mas depois estagna
%     5.2 citar o que a litetatura recomenda
%     5.3. citar que é semelhante a arquitetura VGG
% 6. Uso de regularizadores não se mostrou bom
%     6.1 tanto l1, l2, como l1_l2
% 7. Evidenciar regras utilizadas para criação das camadas de FC
% 8. Dropout

\begin{table}[!htb]
\centering
\caption{Arquiteturas utilizadas}
\label{tbl-arch}
\resizebox{\textwidth}{!}{%
\begin{tabular}{@{}lllllllll@{}}
\toprule
\multicolumn{1}{c}{1} & \multicolumn{1}{c}{2} & \multicolumn{1}{c}{3} & \multicolumn{1}{c}{4} & \multicolumn{1}{c}{5} & \multicolumn{1}{c}{6} & \multicolumn{1}{c}{7} & \multicolumn{1}{c}{8} & \multicolumn{1}{c}{9} \\ \midrule
Conv2D\_(64, 3, 3)    & Conv2D\_(64, 3, 3)    & Conv2D\_(64, 3, 3)    & Conv2D\_(64, 3, 3)    & Conv2D\_(64, 3, 3)    & Conv2D\_(64, 3, 3)    & Conv2D\_(128, 7, 7)   & Conv2D\_(16, 7, 7)    & Conv2D\_(16, 7, 7)    \\
BatchNorm             & BatchNorm             & BatchNorm             & BatchNorm             & BatchNorm             & BatchNorm             & BatchNorm             & BatchNorm             & BatchNorm             \\
Conv2D\_(64, 3, 3)    & Conv2D\_(64, 3, 3)    & Conv2D\_(64, 3, 3)    & Conv2D\_(64, 3, 3)    & Conv2D\_(64, 3, 3)    & Conv2D\_(64, 3, 3)    & Conv2D\_(128, 7, 7)   & Conv2D\_(16, 7, 7)    & Conv2D\_(16, 7, 7)    \\
BatchNorm             & BatchNorm             & BatchNorm             & BatchNorm             & BatchNorm             & BatchNorm             & BatchNorm             & BatchNorm             & BatchNorm             \\
MaxPool\_(3, 3)       & MaxPool\_(3, 3)       & MaxPool\_(3, 3)       & MaxPool\_(3, 3)       & MaxPool\_(3, 3)       & MaxPool\_(3, 3)       & AvgPool\_(3, 3)       & AvgPool\_(3, 3)       & AvgPool\_(3, 3)       \\
Dropout\_50           & Dropout\_50           & Dropout\_50           & Dropout\_50           & Dropout\_50           & Dropout\_50           & Dropout\_50           & Dropout\_50           & Dropout\_50           \\
Conv2D\_(128, 3, 3)   & Conv2D\_(128, 3, 3)   & Conv2D\_(128, 3, 3)   & Conv2D\_(128, 3, 3)   & Conv2D\_(128, 3, 3)   & Conv2D\_(128, 3, 3)   & Conv2D\_(128, 5, 5)   & Conv2D\_(32, 3, 3)    & Conv2D\_(32, 3, 3)    \\
BatchNorm             & BatchNorm             & BatchNorm             & BatchNorm             & BatchNorm             & BatchNorm             & BatchNorm             & BatchNorm             & BatchNorm             \\
Conv2D\_(128, 3, 3)   & Conv2D\_(128, 3, 3)   & Conv2D\_(128, 3, 3)   & Conv2D\_(128, 3, 3)   & Conv2D\_(128, 3, 3)   & Conv2D\_(128, 3, 3)   & Conv2D\_(128, 5, 5)   & Conv2D\_(32, 3, 3)    & Conv2D\_(32, 3, 3)    \\
BatchNorm             & BatchNorm             & BatchNorm             & BatchNorm             & BatchNorm             & BatchNorm             & BatchNorm             & BatchNorm             & BatchNorm             \\
MaxPool\_(3, 3)       & MaxPool\_(3, 3)       & MaxPool\_(3, 3)       & MaxPool\_(3, 3)       & MaxPool\_(3, 3)       & MaxPool\_(3, 3)       & AvgPool\_(3, 3)       & AvgPool\_(3, 3)       & AvgPool\_(3, 3)       \\
Dropout\_50           & Dropout\_50           & Dropout\_50           & Dropout\_50           & Dropout\_50           & Dropout\_50           & Dropout\_50           & Dropout\_50           & Dropout\_50           \\
Conv2D\_(256, 3, 3)   & Conv2D\_(256, 3, 3)   & Conv2D\_(256, 3, 3)   & Conv2D\_(256, 3, 3)   & Conv2D\_(256, 3, 3)   & Conv2D\_(256, 3, 3)   & Conv2D\_(256, 3, 3)   & Conv2D\_(64, 3, 3)    & Conv2D\_(64, 3, 3)    \\
BatchNorm             & BatchNorm             & BatchNorm             & BatchNorm             & BatchNorm             & BatchNorm             & BatchNorm             & BatchNorm             & BatchNorm             \\
Conv2D\_(256, 3, 3)   & Conv2D\_(256, 3, 3)   & Conv2D\_(256, 3, 3)   & Conv2D\_(256, 3, 3)   & Conv2D\_(256, 3, 3)   & Conv2D\_(256, 3, 3)   & Conv2D\_(256, 3, 3)   & Conv2D\_(64, 3, 3)    & Conv2D\_(64, 3, 3)    \\
BatchNorm             & BatchNorm             & BatchNorm             & BatchNorm             & BatchNorm             & BatchNorm             & BatchNorm             & BatchNorm             & BatchNorm             \\
Conv2D\_(256, 3, 3)   & Conv2D\_(256, 3, 3)   & Conv2D\_(256, 3, 3)   & Conv2D\_(256, 3, 3)   & MaxPool\_(3, 3)       & Conv2D\_(256, 3, 3)   & AvgPool\_(3, 3)       & AvgPool\_(3, 3)       & AvgPool\_(3, 3)       \\
BatchNorm             & BatchNorm             & BatchNorm             & BatchNorm             & Dropout\_50           & BatchNorm             & Dropout\_50           & Dropout\_50           & Dropout\_50           \\
MaxPool\_(3, 3)       & MaxPool\_(3, 3)       & MaxPool\_(3, 3)       & MaxPool\_(3, 3)       & Flatten               & MaxPool\_(3, 3)       & Conv2D\_(256, 3, 3)   & Conv2D\_(128, 3, 3)   & Conv2D\_(128, 3, 3)   \\
Dropout\_50           & Dropout\_50           & Dropout\_50           & Dropout\_50           & FC\_(128)             & Dropout\_50           & BatchNorm             & BatchNorm             & BatchNorm             \\
Conv2D\_(512, 3, 3)   & Flatten               & Flatten               & Flatten               & Dropout\_20           & Flatten               & Conv2D\_(256, 3, 3)   & Conv2D\_(128, 3, 3)   & Conv2D\_(128, 3, 3)   \\
BatchNorm             & FC\_(128)             & FC\_(512)             & FC\_(128)             & FC\_(64)              & FC\_(128)             & BatchNorm             & BatchNorm             & BatchNorm             \\
Conv2D\_(512, 3, 3)   & Dropout\_20           & Dropout\_50           & Dropout\_20           & Dropout\_20           & Dropout\_20           & AvgPool\_(3, 3)       & AvgPool\_(3, 3)       & AvgPool\_(3, 3)       \\
BatchNorm             & FC\_(64)              & FC\_(512)             & FC\_(64)              & FC\_(32)              & FC\_(64)              & Dropout\_50           & Dropout\_50           & Dropout\_50           \\
Conv2D\_(512, 3, 3)   & Dropout\_20           & Dropout\_50           & Dropout\_20           & Dropout\_20           & Dropout\_20           & Flatten               & Flatten               & Flatten               \\
BatchNorm             & FC\_(7)               & FC\_(7)               & FC\_(7)               & FC\_(7)               & FC\_(32)              & FC\_(1048)            & FC\_(512)             & FC\_(512)             \\
MaxPool\_(3, 3)       &                       &                       &                       &                       & Dropout\_20           & Dropout\_20           & Dropout\_20           & Dropout\_50           \\
Dropout\_50           &                       &                       &                       &                       & FC\_(7)               & FC\_(512)             & FC\_(512)             & FC\_(512)             \\
Conv2D\_(512, 3, 3)   &                       &                       &                       &                       &                       & Dropout\_20           & Dropout\_20           & Dropout\_20           \\
BatchNorm             &                       &                       &                       &                       &                       & FC\_(7)               & FC\_(7)               & FC\_(7)               \\
Conv2D\_(512, 3, 3)   &                       &                       &                       &                       &                       &                       &                       &                       \\
BatchNorm             &                       &                       &                       &                       &                       &                       &                       &                       \\
Conv2D\_(512, 3, 3)   &                       &                       &                       &                       &                       &                       &                       &                       \\
BatchNorm             &                       &                       &                       &                       &                       &                       &                       &                       \\
MaxPool\_(3, 3)       &                       &                       &                       &                       &                       &                       &                       &                       \\
Dropout\_50           &                       &                       &                       &                       &                       &                       &                       &                       \\
Flatten               &                       &                       &                       &                       &                       &                       &                       &                       \\
FC\_(512)             &                       &                       &                       &                       &                       &                       &                       &                       \\
Dropout\_20           &                       &                       &                       &                       &                       &                       &                       &                       \\
FC\_(512)             &                       &                       &                       &                       &                       &                       &                       &                       \\
Dropout\_20           &                       &                       &                       &                       &                       &                       &                       &                       \\
FC\_(7)               &                       &                       &                       &                       &                       &                       &                       &                       \\ \bottomrule
\end{tabular}%
}
\end{table}
\begin{table}[!htb]
\centering
\caption{F1 Micro das Arquiteturas utilizadas}
\label{tbl:fscore}
\begin{tabular}{@{}cc@{}}
\toprule
Modelo & F1 Micro           \\ \midrule
1      & 0,6898857620507105 \\
2      & 0,6767901922541097 \\
3      & 0,6606297018668152 \\
4      & 0,6798551128448036 \\
5      & 0,6667595430482028 \\
6      & 0,6781833379771524 \\
7      & 0,6949010866536640 \\
8      & 0,6285873502368348 \\
9      & 0,6244079130677069 \\
\toprule
Ensemble & 0,7174700473669546 \\ \bottomrule
\end{tabular}
\end{table}
