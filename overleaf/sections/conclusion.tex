O \textit{Ensemble} de Redes Convolucionais utilizando o modelo \textit{XGBoost} como sistema de votação apresentou bons resultados na tarefa de classificação, mostrando-se competitivo com modelos deste campo de pesquisa. Ressaltando que este resultado está atrelado exclusivamente ao desempenho na tarefa de classificação, e não quanto ao tempo processamento da resposta pelo modelo. Pois, este tempo é variado de acordo com as características do \textit{hardware} utilizado, contudo, é evidente o alto uso de recursos computacionais pelo modelo proposto, visto que este possui em torno de $32$ milhões de parâmetros nos qual a entrada deve ser processada para sua possível classificação. Visto isso, o emprego deste modelo em aplicações deve ser avaliado de forma cuidadosa.

Contudo, o uso deste modelo em aplicações com acesso a recurso de placa gráfica não deve enfrentar problemas com desempenho, visto que somente as camadas convolucionais possuem altas densidades de cálculo, enquanto as camadas de saída são razas, favorecendo o desempenho em relação a tempo de processamento. Além disto, acredita-se que seja possível melhorar a acurácia do modelo proposto, pois, não foi realizada busca exaustiva para escolha dos parâmetros, mas sim métodos empíricos para a definição destes.
