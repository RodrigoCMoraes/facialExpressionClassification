Para executar a avaliação do modelo proposto na tarefa de classificação, aplicou-se os exemplos contidos na partição de teste, de forma sequencial e supervisionada. Onde o rótulo fornecido pelo modelo para cada exemplo foi armazendo como resultado predito, para futura comparação com o rótulo verdadeiro, que é o contido na base dados.

Após o processo de rotulação das amostras de teste pelo modelo, foram fornecidas as informações de dados rotulados, juntamente dos rótulos verdadeiros, a medida Micro \textit{F1 Score}, onde obteve-se como resultado o valor de $71.74$\%. Contudo, os trabalhos relacionados aqui apresentados fazem uso da medida de acurácia como resultados de desempenho, a qual foi obtida pelo mesmo método da medida Micro \textit{F1 Score}, e resultou em $71.74$\%.
